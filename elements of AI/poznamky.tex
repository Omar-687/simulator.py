\documentclass[]{article}

\begin{document}


Pre niekoho ide o umelé formy života, ktoré môžu prekonať ľudskú inteligenciu, zatiaľ čo iní pojmom umelá inteligencia označujú takmer akúkoľvek technológiu spracovania údajov.


uplatnenie c 1.


Autonómne vozidlá si vyžadujú kombináciu mnohých techník umelej inteligencie: \textbf{prehľadávanie a plánovanie najvhodnejšej trasy z bodu od A do bodu B,} \textbf{počítačové videnie na identifikáciu prekážok a rozhodovanie v podmienkach neistoty na zvládanie pohybu v zložitom a dynamickom prostredí}. Každý z týchto aspektov musí dosahovať takmer bezchybnú presnosť, aby sa zabránilo nehodám.

Rovnaké technológie sa používajú aj v iných autonómnych systémoch, ako sú napríklad roboty na doručovanie zásielok, drony či autonómne plavidlá.

Dôsledky: Bezpečnosť cestnej premávky by sa mala v konečnom dôsledku zlepšiť, keďže spoľahlivosť systémov je na vyššej úrovni ako spoľahlivosť človeka, a mali by sa zefektívniť aj logistické reťazce pri preprave tovaru. Ľudia preberajú úlohu dohľadu: dohliadajú na to, čo sa deje, zatiaľ čo o jazdenie sa starajú stroje. Keďže doprava má v našom každodennom živote kľúčové miesto, je pravdepodobné, že na určité dôsledky sme doteraz ešte ani nepomysleli.

uloha c2

new york times uvodna strana ma personalizovane informacie - na zaklade algoritmov umelej inteligencie.

Rovnako ako facebook,instagram


uplatnenie c3 
alebo na rozpoznavanie tvare



napriklad na filtrovanie fotiek vieme pouzit - na vylepsenie vyzoru.


Napríklad pred päťdesiatimi rokmi sa automatické metódy prehľadávania a plánovania považovali za doménu umelej inteligencie. Dnes sa tieto metódy učia všetci študenti počítačovej vedy. Aj niektoré spôsoby spracovania neurčitých informácií začíname chápať tak dobre, že sa pravdepodobne čoskoro presunú z oblasti umelej inteligencie do štatistiky alebo do výpočtov pravdepodobnosti.


Zatiaľ čo pre vás je uchopiť nejaký predmet veľmi jednoduché, pre roboty je to mimoriadne náročné a tejto problematike sa usilovne venuje mnoho vedcov. Medzi nedávne príklady patrí projekt robotického úchopu spoločnosti Google či robotický zberač karfiolu.

Jo
\end{document}

